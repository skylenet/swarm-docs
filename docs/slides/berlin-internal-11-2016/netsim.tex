\documentclass{beamer}

\usepackage[utf8]{inputenc}
\usepackage{default}


\usepackage{multirow} %for aligning stuff in tables

\usepackage{bbding} %for the smiley face

\usepackage{calc} %calculate spacing \widthof

%tikz stuff
\usepackage{tikzsymbols}
\usepackage{tikz}
\usetikzlibrary{fit,arrows,positioning}

  % Keys to support piece-wise uncovering of elements in TikZ pictures:
  % \node[visible on=<2->](foo){Foo}
  % \node[visible on=<{2,4}>](bar){Bar}   % put braces around comma expressions
  %
  % Internally works by setting opacity=0 when invisible, which has the 
  % adavantage (compared to \node<2->(foo){Foo} that the node is always there, hence
  % always consumes space plus that coordinate (foo) is always available.
  %
  % The actual command that implements the invisibility can be overriden
  % by altering the style invisible. For instance \tikzsset{invisible/.style={opacity=0.2}}
  % would dim the "invisible" parts. Alternatively, the color might be set to white, if the
  % output driver does not support transparencies (e.g., PS) 
  %
  \tikzset{
    invisible/.style={opacity=0},
    visible on/.style={alt={#1{}{invisible}}},
    alt/.code args={<#1>#2#3}{%
      \alt<#1>{\pgfkeysalso{#2}}{\pgfkeysalso{#3}} % \pgfkeysalso doesn't change the path
    },
  }
  \tikzset{
    dimmed/.style={opacity=0.2},
    dimmed on/.style={alt={#1{dimmed}{}}},
    alt/.code args={<#1>#2#3}{%
      \alt<#1>{\pgfkeysalso{#2}}{\pgfkeysalso{#3}} % \pgfkeysalso doesn't change the path
    },
  }
\tikzset{
    %Define standard arrow tip
    >=stealth',
    %Define style for boxes
    peer/.style={
           rectangle,
           rounded corners,
           draw=black, thin,
           text width=3.5em,
           minimum height=2em,
           text centered},
    node/.style={
           rectangle,
           rounded corners,
           draw=black, 
           text width=4.5em,
           minimum height=2em,
           text centered,
           fill={rgb:black,1;white,3}},
    chunk/.style={
           rectangle,
           rounded corners,
           draw=black, 
           text width=2.5em,
           minimum height=1em,
           text centered,
           fill=block title bg},
    % Define arrow style
    point/.style={
           ->,
           thick,
           shorten <=2pt,
           shorten >=2pt,}
every node/.style={align=center}           
}

\newcommand{\wholeslide}[2][]{
\begin{frame}{#1}
% \transboxout<1>[duration=0.5]
\setbeamercolor{bgcolor}{fg=black,bg=white}
\begin{tikzpicture}[overlay, remember picture]
\node[anchor=center] at (current page.center) {
  \begin{beamercolorbox}[center]{bgcolor}
     #2
  \end{beamercolorbox}};
\end{tikzpicture}

\end{frame}
}

\newcommand{\blankslide}[2][]{
\begin{frame}[plain]{#1}
% \transboxout<1>[duration=0.5]
\setbeamercolor{bgcolor}{fg=black,bg=white}
\begin{tikzpicture}[overlay, remember picture]
\node[anchor=center] at (current page.center) {
  \begin{beamercolorbox}[center]{bgcolor}
     #2
  \end{beamercolorbox}};
\end{tikzpicture}

\end{frame}
}

\newenvironment<>{varblock}[2][.9\textwidth]{%
  \setlength{\textwidth}{#1}
  \begin{actionenv}#3%
    \def\insertblocktitle{#2}%
    \par%
    \usebeamertemplate{block begin}}
  {\par%
    \usebeamertemplate{block end}%
  \end{actionenv}}

\newlength{\mywidth}
\newcommand{\blockslide}[3][]{
\settowidth{\mywidth}{#3}
\begin{frame}[c]{#1}
\begin{center}
\begin{minipage}{1.1\mywidth}
 \begin{varblock}[1.1\mywidth]{#2}
  #3
 \end{varblock}
\end{minipage}
\end{center}
\end{frame}
}

\mode<presentation>{
\usetheme{Warsaw}\usecolortheme{crane}
\setbeamertemplate{items}[square]
\setbeamertemplate{section in toc}[square]
\setbeamertemplate{subsection in toc}[square]
% \setbeamertemplate{subsection in toc}[subsections numbered]
\setbeamertemplate{subsubsection in toc}[square]
% \setbeamercolor{items}{fg=black,bg=yellow}
\usebeamercolor{block title}
\definecolor{block title bg}{named}{bg}
\setbeamercolor{item projected}{fg=black,bg=block title bg}
\setbeamercolor{itemize item}{fg=block title bg,bg=block title bg}
}







\title{devp2p network simulation, testing and monitoring framework}
\author{Viktor Trón}

\AtBeginSection[]
{
\begin{frame}<beamer>
\frametitle{Outline}
\tableofcontents[currentsection,sectionstyle=show/shaded,subsectionstyle=show/show/hide,subsubsectionstyle=show/show/show/hide]
\end{frame}
}

\begin{document}

\begin{frame}
 \titlepage
\end{frame}

\begin{frame}
\begin{block}{}
\begin{itemize}
\item motivation and context
\item scope and status
\item features and design
\item demo and code dive
\end{itemize}
\end{block}
\end{frame}

\begin{section}{motivation and context}


\begin{frame}{}
\begin{block}{swarm network layer rewrite}
 \begin{itemize}
    \item syncing rewrite
    \item testing and fixing edge cases with kademlia overlay topology bootstrapping
    \item integration whisper/pss in status/raiden
    \item testing smash/crash proof of custody
    \item testing receipt passing, storage insurance logic, litigation
    \item retrieval using erasure coded chunk, scan and repair for erasure coding
\end{itemize}
\end{block}
\end{frame}


\plainblockslide{}{\textsc{pss}: \textbf{p}ostal \textbf{s}ervices \textbf{s}uite (bzz whispered)}

\end{section}

\begin{section}{scope and status}


\begin{frame}{}
\begin{block}{status}
 \begin{itemize}
    \item basic skeleton
    \item inprocess adapter
    \item remote in the making
    \item a bunch of volunteers contributing
    \item whisper simulation
\end{itemize}
\end{block}
\end{frame}

\end{section}

\begin{section}{features and design}

\begin{frame}{}
\begin{block}{network model and basic concepts}
 \begin{itemize}
    \item generic network model abstract out underlay connectivity
    \item nodes and connections
    \item trigger events: startup/shutdown node, connect node to another, etc
    \item underlay adapters
    \item output events
    \item journals, snapshots
    \item controllers
    \item test drivers
\end{itemize}
\end{block}
\end{frame}

\begin{frame}{}
\begin{block}{underlay adapters for network/transport types}
 \begin{itemize}
    \item plain rlpx
    \item rlpx with output event reporting for test docker container
    \item inprocess adaptor for simulations using simulated message pipe
    \item inprocess adaptor using p2p server instance (and discovery)
    \item remote node using ssh (kubernetes)
\end{itemize}
\end{block}
\end{frame}


\begin{frame}{}
\begin{block}{events and the journal}
 \begin{itemize}

\item  event logged as entries in a journal, can be replayed (on a different network type)
\item  connectivity journal, message passing journal
\item  dynamic network visualisation using the REST Api (example with mocked journal)
\item  journal can be replayed to get a snapshot (connectivity or storage/msg passing)
\item  snapshots are portable across network types so can setup/bootstrap simulations
\end{itemize}
\end{block}
\end{frame}


\begin{frame}{}
\begin{block}{events control}
 \begin{itemize}
\item  run history as a journal subscribing to events
\item  reset, record
\item  Read: without constraints, it reads all the events in the buffer,
\item  TimedRead: replay with acceleration (for simulation/visualisation) faithful to intervals
\item  Scheduler: play waiting for actual or fake event time
\item  UI to manually construct test setups
\item  mockers mock event generation

\end{itemize}
\end{block}
\end{frame}

\begin{frame}{}
\begin{block}{test drivers and benchmark tools}
\begin{itemize}

\item  snapshots are used to setup initial states
\item  journal can be used to schedule drive network scenarios
\item  trigger/expect testing pattern for p2p message exchange for protocol unit testing
\item  trigger/expect pattern for unicast and multicast overlay routing (how pss and whisper will be tested)
\item  suite of controllers provide a resourceful REST API to query and dynamically change the network, trigger events and query metrics (using the metrics package)
\item  The messengers can be used to meter bandwidth, throughput
\item  latency benchmarks could be simulated using parameters based on a real network measurements

\end{itemize}
\end{block}
\end{frame}






\end{section}

\begin{section}{demo and code dive}

\begin{frame}{}
\begin{block}{UX}
 \begin{itemize}
\item  connectivity and message passing visualisations using cytoscape.js http://js.cytoscape.org/
\item  cluster management and monitoring dashboard
\item  UI for simulation and aggregate stats
\item dynamic connectivity
\end{itemize}
\end{block}

\begin{block}{backend}
 \begin{itemize}
\item  local http server exposing a resourceful REST API
\item  hierarchical embedded resuorce controllers
\item REST API routes map to resource controllers and CRUD actions
\item dynamic network visualisation
\end{itemize}
\end{block}
\end{frame}


\end{section}
\end{document}
